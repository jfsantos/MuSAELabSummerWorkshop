
% Default to the notebook output style

    


% Inherit from the specified cell style.




    
\documentclass{article}

    
    
    \usepackage{graphicx} % Used to insert images
    \usepackage{adjustbox} % Used to constrain images to a maximum size 
    \usepackage{color} % Allow colors to be defined
    \usepackage{enumerate} % Needed for markdown enumerations to work
    \usepackage{geometry} % Used to adjust the document margins
    \usepackage{amsmath} % Equations
    \usepackage{amssymb} % Equations
    \usepackage[mathletters]{ucs} % Extended unicode (utf-8) support
    \usepackage[utf8x]{inputenc} % Allow utf-8 characters in the tex document
    \usepackage{fancyvrb} % verbatim replacement that allows latex
    \usepackage{grffile} % extends the file name processing of package graphics 
                         % to support a larger range 
    % The hyperref package gives us a pdf with properly built
    % internal navigation ('pdf bookmarks' for the table of contents,
    % internal cross-reference links, web links for URLs, etc.)
    \usepackage{hyperref}
    \usepackage{longtable} % longtable support required by pandoc >1.10
    \usepackage{booktabs}  % table support for pandoc > 1.12.2
    

    
    
    \definecolor{orange}{cmyk}{0,0.4,0.8,0.2}
    \definecolor{darkorange}{rgb}{.71,0.21,0.01}
    \definecolor{darkgreen}{rgb}{.12,.54,.11}
    \definecolor{myteal}{rgb}{.26, .44, .56}
    \definecolor{gray}{gray}{0.45}
    \definecolor{lightgray}{gray}{.95}
    \definecolor{mediumgray}{gray}{.8}
    \definecolor{inputbackground}{rgb}{.95, .95, .85}
    \definecolor{outputbackground}{rgb}{.95, .95, .95}
    \definecolor{traceback}{rgb}{1, .95, .95}
    % ansi colors
    \definecolor{red}{rgb}{.6,0,0}
    \definecolor{green}{rgb}{0,.65,0}
    \definecolor{brown}{rgb}{0.6,0.6,0}
    \definecolor{blue}{rgb}{0,.145,.698}
    \definecolor{purple}{rgb}{.698,.145,.698}
    \definecolor{cyan}{rgb}{0,.698,.698}
    \definecolor{lightgray}{gray}{0.5}
    
    % bright ansi colors
    \definecolor{darkgray}{gray}{0.25}
    \definecolor{lightred}{rgb}{1.0,0.39,0.28}
    \definecolor{lightgreen}{rgb}{0.48,0.99,0.0}
    \definecolor{lightblue}{rgb}{0.53,0.81,0.92}
    \definecolor{lightpurple}{rgb}{0.87,0.63,0.87}
    \definecolor{lightcyan}{rgb}{0.5,1.0,0.83}
    
    % commands and environments needed by pandoc snippets
    % extracted from the output of `pandoc -s`
    \DefineVerbatimEnvironment{Highlighting}{Verbatim}{commandchars=\\\{\}}
    % Add ',fontsize=\small' for more characters per line
    \newenvironment{Shaded}{}{}
    \newcommand{\KeywordTok}[1]{\textcolor[rgb]{0.00,0.44,0.13}{\textbf{{#1}}}}
    \newcommand{\DataTypeTok}[1]{\textcolor[rgb]{0.56,0.13,0.00}{{#1}}}
    \newcommand{\DecValTok}[1]{\textcolor[rgb]{0.25,0.63,0.44}{{#1}}}
    \newcommand{\BaseNTok}[1]{\textcolor[rgb]{0.25,0.63,0.44}{{#1}}}
    \newcommand{\FloatTok}[1]{\textcolor[rgb]{0.25,0.63,0.44}{{#1}}}
    \newcommand{\CharTok}[1]{\textcolor[rgb]{0.25,0.44,0.63}{{#1}}}
    \newcommand{\StringTok}[1]{\textcolor[rgb]{0.25,0.44,0.63}{{#1}}}
    \newcommand{\CommentTok}[1]{\textcolor[rgb]{0.38,0.63,0.69}{\textit{{#1}}}}
    \newcommand{\OtherTok}[1]{\textcolor[rgb]{0.00,0.44,0.13}{{#1}}}
    \newcommand{\AlertTok}[1]{\textcolor[rgb]{1.00,0.00,0.00}{\textbf{{#1}}}}
    \newcommand{\FunctionTok}[1]{\textcolor[rgb]{0.02,0.16,0.49}{{#1}}}
    \newcommand{\RegionMarkerTok}[1]{{#1}}
    \newcommand{\ErrorTok}[1]{\textcolor[rgb]{1.00,0.00,0.00}{\textbf{{#1}}}}
    \newcommand{\NormalTok}[1]{{#1}}
    
    % Define a nice break command that doesn't care if a line doesn't already
    % exist.
    \def\br{\hspace*{\fill} \\* }
    % Math Jax compatability definitions
    \def\gt{>}
    \def\lt{<}
    % Document parameters
    \title{Section1\_2-Programming-with-Python}
    
    
    

    % Pygments definitions
    
\makeatletter
\def\PY@reset{\let\PY@it=\relax \let\PY@bf=\relax%
    \let\PY@ul=\relax \let\PY@tc=\relax%
    \let\PY@bc=\relax \let\PY@ff=\relax}
\def\PY@tok#1{\csname PY@tok@#1\endcsname}
\def\PY@toks#1+{\ifx\relax#1\empty\else%
    \PY@tok{#1}\expandafter\PY@toks\fi}
\def\PY@do#1{\PY@bc{\PY@tc{\PY@ul{%
    \PY@it{\PY@bf{\PY@ff{#1}}}}}}}
\def\PY#1#2{\PY@reset\PY@toks#1+\relax+\PY@do{#2}}

\expandafter\def\csname PY@tok@gd\endcsname{\def\PY@tc##1{\textcolor[rgb]{0.63,0.00,0.00}{##1}}}
\expandafter\def\csname PY@tok@gu\endcsname{\let\PY@bf=\textbf\def\PY@tc##1{\textcolor[rgb]{0.50,0.00,0.50}{##1}}}
\expandafter\def\csname PY@tok@gt\endcsname{\def\PY@tc##1{\textcolor[rgb]{0.00,0.27,0.87}{##1}}}
\expandafter\def\csname PY@tok@gs\endcsname{\let\PY@bf=\textbf}
\expandafter\def\csname PY@tok@gr\endcsname{\def\PY@tc##1{\textcolor[rgb]{1.00,0.00,0.00}{##1}}}
\expandafter\def\csname PY@tok@cm\endcsname{\let\PY@it=\textit\def\PY@tc##1{\textcolor[rgb]{0.25,0.50,0.50}{##1}}}
\expandafter\def\csname PY@tok@vg\endcsname{\def\PY@tc##1{\textcolor[rgb]{0.10,0.09,0.49}{##1}}}
\expandafter\def\csname PY@tok@m\endcsname{\def\PY@tc##1{\textcolor[rgb]{0.40,0.40,0.40}{##1}}}
\expandafter\def\csname PY@tok@mh\endcsname{\def\PY@tc##1{\textcolor[rgb]{0.40,0.40,0.40}{##1}}}
\expandafter\def\csname PY@tok@go\endcsname{\def\PY@tc##1{\textcolor[rgb]{0.53,0.53,0.53}{##1}}}
\expandafter\def\csname PY@tok@ge\endcsname{\let\PY@it=\textit}
\expandafter\def\csname PY@tok@vc\endcsname{\def\PY@tc##1{\textcolor[rgb]{0.10,0.09,0.49}{##1}}}
\expandafter\def\csname PY@tok@il\endcsname{\def\PY@tc##1{\textcolor[rgb]{0.40,0.40,0.40}{##1}}}
\expandafter\def\csname PY@tok@cs\endcsname{\let\PY@it=\textit\def\PY@tc##1{\textcolor[rgb]{0.25,0.50,0.50}{##1}}}
\expandafter\def\csname PY@tok@cp\endcsname{\def\PY@tc##1{\textcolor[rgb]{0.74,0.48,0.00}{##1}}}
\expandafter\def\csname PY@tok@gi\endcsname{\def\PY@tc##1{\textcolor[rgb]{0.00,0.63,0.00}{##1}}}
\expandafter\def\csname PY@tok@gh\endcsname{\let\PY@bf=\textbf\def\PY@tc##1{\textcolor[rgb]{0.00,0.00,0.50}{##1}}}
\expandafter\def\csname PY@tok@ni\endcsname{\let\PY@bf=\textbf\def\PY@tc##1{\textcolor[rgb]{0.60,0.60,0.60}{##1}}}
\expandafter\def\csname PY@tok@nl\endcsname{\def\PY@tc##1{\textcolor[rgb]{0.63,0.63,0.00}{##1}}}
\expandafter\def\csname PY@tok@nn\endcsname{\let\PY@bf=\textbf\def\PY@tc##1{\textcolor[rgb]{0.00,0.00,1.00}{##1}}}
\expandafter\def\csname PY@tok@no\endcsname{\def\PY@tc##1{\textcolor[rgb]{0.53,0.00,0.00}{##1}}}
\expandafter\def\csname PY@tok@na\endcsname{\def\PY@tc##1{\textcolor[rgb]{0.49,0.56,0.16}{##1}}}
\expandafter\def\csname PY@tok@nb\endcsname{\def\PY@tc##1{\textcolor[rgb]{0.00,0.50,0.00}{##1}}}
\expandafter\def\csname PY@tok@nc\endcsname{\let\PY@bf=\textbf\def\PY@tc##1{\textcolor[rgb]{0.00,0.00,1.00}{##1}}}
\expandafter\def\csname PY@tok@nd\endcsname{\def\PY@tc##1{\textcolor[rgb]{0.67,0.13,1.00}{##1}}}
\expandafter\def\csname PY@tok@ne\endcsname{\let\PY@bf=\textbf\def\PY@tc##1{\textcolor[rgb]{0.82,0.25,0.23}{##1}}}
\expandafter\def\csname PY@tok@nf\endcsname{\def\PY@tc##1{\textcolor[rgb]{0.00,0.00,1.00}{##1}}}
\expandafter\def\csname PY@tok@si\endcsname{\let\PY@bf=\textbf\def\PY@tc##1{\textcolor[rgb]{0.73,0.40,0.53}{##1}}}
\expandafter\def\csname PY@tok@s2\endcsname{\def\PY@tc##1{\textcolor[rgb]{0.73,0.13,0.13}{##1}}}
\expandafter\def\csname PY@tok@vi\endcsname{\def\PY@tc##1{\textcolor[rgb]{0.10,0.09,0.49}{##1}}}
\expandafter\def\csname PY@tok@nt\endcsname{\let\PY@bf=\textbf\def\PY@tc##1{\textcolor[rgb]{0.00,0.50,0.00}{##1}}}
\expandafter\def\csname PY@tok@nv\endcsname{\def\PY@tc##1{\textcolor[rgb]{0.10,0.09,0.49}{##1}}}
\expandafter\def\csname PY@tok@s1\endcsname{\def\PY@tc##1{\textcolor[rgb]{0.73,0.13,0.13}{##1}}}
\expandafter\def\csname PY@tok@sh\endcsname{\def\PY@tc##1{\textcolor[rgb]{0.73,0.13,0.13}{##1}}}
\expandafter\def\csname PY@tok@sc\endcsname{\def\PY@tc##1{\textcolor[rgb]{0.73,0.13,0.13}{##1}}}
\expandafter\def\csname PY@tok@sx\endcsname{\def\PY@tc##1{\textcolor[rgb]{0.00,0.50,0.00}{##1}}}
\expandafter\def\csname PY@tok@bp\endcsname{\def\PY@tc##1{\textcolor[rgb]{0.00,0.50,0.00}{##1}}}
\expandafter\def\csname PY@tok@c1\endcsname{\let\PY@it=\textit\def\PY@tc##1{\textcolor[rgb]{0.25,0.50,0.50}{##1}}}
\expandafter\def\csname PY@tok@kc\endcsname{\let\PY@bf=\textbf\def\PY@tc##1{\textcolor[rgb]{0.00,0.50,0.00}{##1}}}
\expandafter\def\csname PY@tok@c\endcsname{\let\PY@it=\textit\def\PY@tc##1{\textcolor[rgb]{0.25,0.50,0.50}{##1}}}
\expandafter\def\csname PY@tok@mf\endcsname{\def\PY@tc##1{\textcolor[rgb]{0.40,0.40,0.40}{##1}}}
\expandafter\def\csname PY@tok@err\endcsname{\def\PY@bc##1{\setlength{\fboxsep}{0pt}\fcolorbox[rgb]{1.00,0.00,0.00}{1,1,1}{\strut ##1}}}
\expandafter\def\csname PY@tok@kd\endcsname{\let\PY@bf=\textbf\def\PY@tc##1{\textcolor[rgb]{0.00,0.50,0.00}{##1}}}
\expandafter\def\csname PY@tok@ss\endcsname{\def\PY@tc##1{\textcolor[rgb]{0.10,0.09,0.49}{##1}}}
\expandafter\def\csname PY@tok@sr\endcsname{\def\PY@tc##1{\textcolor[rgb]{0.73,0.40,0.53}{##1}}}
\expandafter\def\csname PY@tok@mo\endcsname{\def\PY@tc##1{\textcolor[rgb]{0.40,0.40,0.40}{##1}}}
\expandafter\def\csname PY@tok@kn\endcsname{\let\PY@bf=\textbf\def\PY@tc##1{\textcolor[rgb]{0.00,0.50,0.00}{##1}}}
\expandafter\def\csname PY@tok@mi\endcsname{\def\PY@tc##1{\textcolor[rgb]{0.40,0.40,0.40}{##1}}}
\expandafter\def\csname PY@tok@gp\endcsname{\let\PY@bf=\textbf\def\PY@tc##1{\textcolor[rgb]{0.00,0.00,0.50}{##1}}}
\expandafter\def\csname PY@tok@o\endcsname{\def\PY@tc##1{\textcolor[rgb]{0.40,0.40,0.40}{##1}}}
\expandafter\def\csname PY@tok@kr\endcsname{\let\PY@bf=\textbf\def\PY@tc##1{\textcolor[rgb]{0.00,0.50,0.00}{##1}}}
\expandafter\def\csname PY@tok@s\endcsname{\def\PY@tc##1{\textcolor[rgb]{0.73,0.13,0.13}{##1}}}
\expandafter\def\csname PY@tok@kp\endcsname{\def\PY@tc##1{\textcolor[rgb]{0.00,0.50,0.00}{##1}}}
\expandafter\def\csname PY@tok@w\endcsname{\def\PY@tc##1{\textcolor[rgb]{0.73,0.73,0.73}{##1}}}
\expandafter\def\csname PY@tok@kt\endcsname{\def\PY@tc##1{\textcolor[rgb]{0.69,0.00,0.25}{##1}}}
\expandafter\def\csname PY@tok@ow\endcsname{\let\PY@bf=\textbf\def\PY@tc##1{\textcolor[rgb]{0.67,0.13,1.00}{##1}}}
\expandafter\def\csname PY@tok@sb\endcsname{\def\PY@tc##1{\textcolor[rgb]{0.73,0.13,0.13}{##1}}}
\expandafter\def\csname PY@tok@k\endcsname{\let\PY@bf=\textbf\def\PY@tc##1{\textcolor[rgb]{0.00,0.50,0.00}{##1}}}
\expandafter\def\csname PY@tok@se\endcsname{\let\PY@bf=\textbf\def\PY@tc##1{\textcolor[rgb]{0.73,0.40,0.13}{##1}}}
\expandafter\def\csname PY@tok@sd\endcsname{\let\PY@it=\textit\def\PY@tc##1{\textcolor[rgb]{0.73,0.13,0.13}{##1}}}

\def\PYZbs{\char`\\}
\def\PYZus{\char`\_}
\def\PYZob{\char`\{}
\def\PYZcb{\char`\}}
\def\PYZca{\char`\^}
\def\PYZam{\char`\&}
\def\PYZlt{\char`\<}
\def\PYZgt{\char`\>}
\def\PYZsh{\char`\#}
\def\PYZpc{\char`\%}
\def\PYZdl{\char`\$}
\def\PYZhy{\char`\-}
\def\PYZsq{\char`\'}
\def\PYZdq{\char`\"}
\def\PYZti{\char`\~}
% for compatibility with earlier versions
\def\PYZat{@}
\def\PYZlb{[}
\def\PYZrb{]}
\makeatother


    % Exact colors from NB
    \definecolor{incolor}{rgb}{0.0, 0.0, 0.5}
    \definecolor{outcolor}{rgb}{0.545, 0.0, 0.0}



    
    % Prevent overflowing lines due to hard-to-break entities
    \sloppy 
    % Setup hyperref package
    \hypersetup{
      breaklinks=true,  % so long urls are correctly broken across lines
      colorlinks=true,
      urlcolor=blue,
      linkcolor=darkorange,
      citecolor=darkgreen,
      }
    % Slightly bigger margins than the latex defaults
    
    \geometry{verbose,tmargin=1in,bmargin=1in,lmargin=1in,rmargin=1in}
    
    

    \begin{document}
    
    
    \maketitle
    
    

    
    \section{Programming with Python}\label{programming-with-python}

    \subsection{Control Flow Statements}\label{control-flow-statements}

The control flow of a program determines the order in which lines of
code are executed. All else being equal, Python code is executed
linearly, in the order that lines appear in the program. However, all is
not usually equal, and so the appropriate control flow is frequently
specified with the help of control flow statements. These include loops,
conditional statements and calls to functions. Let's look at a few of
these here.

\textbf{\emph{for statements}}\\One way to repeatedly execute a block of
statements (\emph{i.e.} loop) is to use a \texttt{for} statement. These
statements iterate over the number of elements in a specified sequence,
according to the following syntax:

    \begin{Verbatim}[commandchars=\\\{\}]
{\color{incolor}In [{\color{incolor}1}]:} \PY{k}{for} \PY{n}{letter} \PY{o+ow}{in} \PY{l+s}{\PYZsq{}}\PY{l+s}{ciao}\PY{l+s}{\PYZsq{}}\PY{p}{:}
            \PY{k}{print}\PY{p}{(}\PY{l+s}{\PYZsq{}}\PY{l+s}{give me a \PYZob{}0\PYZcb{}}\PY{l+s}{\PYZsq{}}\PY{o}{.}\PY{n}{format}\PY{p}{(}\PY{n}{letter}\PY{o}{.}\PY{n}{upper}\PY{p}{(}\PY{p}{)}\PY{p}{)}\PY{p}{)}
\end{Verbatim}

    \begin{Verbatim}[commandchars=\\\{\}]
give me a C
give me a I
give me a A
give me a O
    \end{Verbatim}

    Recall that strings are simply regarded as sequences of characters.
Hence, the above \texttt{for} statement loops over each letter,
converting each to upper case with the \texttt{upper()} method and
printing it.

Similarly, as shown in the introduction, list comprehensions may be
constructed using \texttt{for} statements:

    \begin{Verbatim}[commandchars=\\\{\}]
{\color{incolor}In [{\color{incolor}2}]:} \PY{p}{[}\PY{n}{i}\PY{o}{*}\PY{o}{*}\PY{l+m+mi}{2} \PY{k}{for} \PY{n}{i} \PY{o+ow}{in} \PY{n+nb}{range}\PY{p}{(}\PY{l+m+mi}{10}\PY{p}{)}\PY{p}{]}
\end{Verbatim}

            \begin{Verbatim}[commandchars=\\\{\}]
{\color{outcolor}Out[{\color{outcolor}2}]:} [0, 1, 4, 9, 16, 25, 36, 49, 64, 81]
\end{Verbatim}
        
    Here, the expression loops over \texttt{range(10)} -- the sequence from
0 to 9 -- and squares each before placing it in the returned list.

\textbf{\emph{if statements}}\\As the name implies, \texttt{if}
statements execute particular sections of code depending on some tested
condition. For example, to code an absolute value function, one might
employ conditional statements:

    \begin{Verbatim}[commandchars=\\\{\}]
{\color{incolor}In [{\color{incolor}3}]:} \PY{k}{def} \PY{n+nf}{absval}\PY{p}{(}\PY{n}{some\PYZus{}list}\PY{p}{)}\PY{p}{:}
        
            \PY{c}{\PYZsh{} Create empty list}
            \PY{n}{absolutes} \PY{o}{=} \PY{p}{[}\PY{p}{]}    
        
            \PY{c}{\PYZsh{} Loop over elements in some\PYZus{}list}
            \PY{k}{for} \PY{n}{value} \PY{o+ow}{in} \PY{n}{some\PYZus{}list}\PY{p}{:}
        
                \PY{c}{\PYZsh{} Conditional statement}
                \PY{k}{if} \PY{n}{value}\PY{o}{\PYZlt{}}\PY{l+m+mi}{0}\PY{p}{:}
                    \PY{c}{\PYZsh{} Negative value}
                    \PY{n}{absolutes}\PY{o}{.}\PY{n}{append}\PY{p}{(}\PY{o}{\PYZhy{}}\PY{n}{value}\PY{p}{)}
        
                \PY{k}{else}\PY{p}{:}
                    \PY{c}{\PYZsh{} Positive value}
                    \PY{n}{absolutes}\PY{o}{.}\PY{n}{append}\PY{p}{(}\PY{n}{value}\PY{p}{)}
            
            \PY{k}{return} \PY{n}{absolutes} 
\end{Verbatim}

    Here, each value in \texttt{some\_list} is tested for the condition that
it is negative, in which case it is multiplied by -1, otherwise it is
appended as-is. For conditions that have more than two possible values,
the elif statement can be used:

    \begin{Verbatim}[commandchars=\\\{\}]
{\color{incolor}In [{\color{incolor}4}]:} \PY{n}{x} \PY{o}{=} \PY{l+m+mi}{5}
        \PY{k}{if} \PY{n}{x} \PY{o}{\PYZlt{}} \PY{l+m+mi}{0}\PY{p}{:}
            \PY{k}{print}\PY{p}{(}\PY{l+s}{\PYZsq{}}\PY{l+s}{x is negative}\PY{l+s}{\PYZsq{}}\PY{p}{)}
        \PY{k}{elif} \PY{n}{x} \PY{o}{\PYZpc{}} \PY{l+m+mi}{2}\PY{p}{:}
            \PY{k}{print}\PY{p}{(}\PY{l+s}{\PYZsq{}}\PY{l+s}{x is positive and odd}\PY{l+s}{\PYZsq{}}\PY{p}{)}
        \PY{k}{else}\PY{p}{:}
            \PY{k}{print}\PY{p}{(}\PY{l+s}{\PYZsq{}}\PY{l+s}{x is even and non\PYZhy{}negative}\PY{l+s}{\PYZsq{}}\PY{p}{)}
\end{Verbatim}

    \begin{Verbatim}[commandchars=\\\{\}]
x is positive and odd
    \end{Verbatim}

    \textbf{\emph{while statements}}

A different type of conditional loop is provided by the \texttt{while}
statement. Rather than iterating a specified number of times, according
to a given sequence, \texttt{while} executes its block of code
repeatedly, until its condition is no longer true. For example, suppose
we want to sample from a truncated normal distribution, where we are
only interested in positive-valued samples. The following function is
one solution:

    \begin{Verbatim}[commandchars=\\\{\}]
{\color{incolor}In [{\color{incolor}5}]:} \PY{c}{\PYZsh{} Import function}
        \PY{k+kn}{from} \PY{n+nn}{numpy.random} \PY{k+kn}{import} \PY{n}{normal}
        
        \PY{k}{def} \PY{n+nf}{truncated\PYZus{}normals}\PY{p}{(}\PY{n}{how\PYZus{}many}\PY{p}{,} \PY{n}{l}\PY{p}{)}\PY{p}{:}
        
            \PY{c}{\PYZsh{} Create empty list}
            \PY{n}{values} \PY{o}{=} \PY{p}{[}\PY{p}{]}
        
            \PY{c}{\PYZsh{} Loop until we have specified number of samples}
            \PY{k}{while} \PY{p}{(}\PY{n+nb}{len}\PY{p}{(}\PY{n}{values}\PY{p}{)} \PY{o}{\PYZlt{}} \PY{n}{how\PYZus{}many}\PY{p}{)}\PY{p}{:}
        
                \PY{c}{\PYZsh{} Sample from standard normal}
                \PY{n}{x} \PY{o}{=} \PY{n}{normal}\PY{p}{(}\PY{l+m+mi}{0}\PY{p}{,}\PY{l+m+mi}{1}\PY{p}{)}
        
                \PY{c}{\PYZsh{} Append if not truncateed}
                \PY{k}{if} \PY{n}{x} \PY{o}{\PYZgt{}} \PY{n}{l}\PY{p}{:} \PY{n}{values}\PY{o}{.}\PY{n}{append}\PY{p}{(}\PY{n}{x}\PY{p}{)}
        
            \PY{k}{return} \PY{n}{values}    
\end{Verbatim}

    \begin{Verbatim}[commandchars=\\\{\}]
{\color{incolor}In [{\color{incolor}6}]:} \PY{n}{truncated\PYZus{}normals}\PY{p}{(}\PY{l+m+mi}{15}\PY{p}{,} \PY{l+m+mi}{0}\PY{p}{)}
\end{Verbatim}

            \begin{Verbatim}[commandchars=\\\{\}]
{\color{outcolor}Out[{\color{outcolor}6}]:} [1.1738372661791174,
         0.45099478603456566,
         1.6317748177330285,
         0.41275970336306655,
         1.4035699237984376,
         0.45342045186201724,
         0.795107166506508,
         1.803591568428664,
         0.14792870498175092,
         0.31721171897525846,
         0.9227994109843113,
         1.0430789423408247,
         2.2330729023818003,
         1.7023236403370337,
         0.16640359787626927]
\end{Verbatim}
        
    This function iteratively samples from a standard normal distribution,
and appends it to the output array if it is positive, stopping to return
the array once the specified number of values have been added.

Obviously, the body of the \texttt{while} statement should contain code
that eventually renders the condition false, otherwise the loop will
never end! An exception to this is if the body of the statement contains
a \texttt{break} or \texttt{return} statement; in either case, the loop
will be interrupted.

    \subsection{Error Handling}\label{error-handling}

Inevitably, some code you write will generate errors, at least in some
situations. Unless we explicitly anticipate and \textbf{handle} these
errors, they will cause your code to halt (sometimes this is a good
thing!). Errors are handled using \texttt{try/except} blocks.

If code executed in the \texttt{try} block generates an error, code
execution moves to the \texttt{except} block. If the exception that is
specified corresponsd to that which has been raised, the code in the
\texttt{except} block is executed before continuing; otherwise, the
exception is carried out and the code is halted.

    \begin{Verbatim}[commandchars=\\\{\}]
{\color{incolor}In [{\color{incolor}7}]:} \PY{n}{absval}\PY{p}{(}\PY{o}{\PYZhy{}}\PY{l+m+mi}{5}\PY{p}{)}
\end{Verbatim}

    \begin{Verbatim}[commandchars=\\\{\}]

        ---------------------------------------------------------------------------
    TypeError                                 Traceback (most recent call last)

        <ipython-input-7-241bbde36660> in <module>()
    ----> 1 absval(-5)
    

        <ipython-input-3-e94eca0d977f> in absval(some\_list)
          5 
          6     \# Loop over elements in some\_list
    ----> 7     for value in some\_list:
          8 
          9         \# Conditional statement


        TypeError: 'int' object is not iterable

    \end{Verbatim}

    In the call to \texttt{absval}, we passed a single negative integer,
whereas the function expects some sort of iterable data structure. Other
than changing the function itself, we can avoid this error using
exception handling.

    \begin{Verbatim}[commandchars=\\\{\}]
{\color{incolor}In [{\color{incolor}8}]:} \PY{n}{x} \PY{o}{=} \PY{o}{\PYZhy{}}\PY{l+m+mi}{5}
        \PY{k}{try}\PY{p}{:}
            \PY{k}{print}\PY{p}{(}\PY{n}{absval}\PY{p}{(}\PY{n}{x}\PY{p}{)}\PY{p}{)}
        \PY{k}{except} \PY{n+ne}{TypeError}\PY{p}{:}
            \PY{k}{print}\PY{p}{(}\PY{l+s}{\PYZsq{}}\PY{l+s}{The argument to absval must be iterable!}\PY{l+s}{\PYZsq{}}\PY{p}{)}
\end{Verbatim}

    \begin{Verbatim}[commandchars=\\\{\}]
The argument to absval must be iterable!
    \end{Verbatim}

    \begin{Verbatim}[commandchars=\\\{\}]
{\color{incolor}In [{\color{incolor}9}]:} \PY{n}{x} \PY{o}{=} \PY{o}{\PYZhy{}}\PY{l+m+mi}{5}
        \PY{k}{try}\PY{p}{:}
            \PY{k}{print}\PY{p}{(}\PY{n}{absval}\PY{p}{(}\PY{n}{x}\PY{p}{)}\PY{p}{)}
        \PY{k}{except} \PY{n+ne}{TypeError}\PY{p}{:}
            \PY{k}{print}\PY{p}{(}\PY{n}{absval}\PY{p}{(}\PY{p}{[}\PY{n}{x}\PY{p}{]}\PY{p}{)}\PY{p}{)}
\end{Verbatim}

    \begin{Verbatim}[commandchars=\\\{\}]
[5]
    \end{Verbatim}

    We can raise exceptions manually by using the \texttt{raise} expression.

    \begin{Verbatim}[commandchars=\\\{\}]
{\color{incolor}In [{\color{incolor}10}]:} \PY{k}{raise} \PY{n+ne}{ValueError}\PY{p}{(}\PY{l+s}{\PYZsq{}}\PY{l+s}{This is the wrong value}\PY{l+s}{\PYZsq{}}\PY{p}{)}
\end{Verbatim}

    \begin{Verbatim}[commandchars=\\\{\}]

        ---------------------------------------------------------------------------
    ValueError                                Traceback (most recent call last)

        <ipython-input-10-7b8b3dfef9b4> in <module>()
    ----> 1 raise ValueError('This is the wrong value')
    

        ValueError: This is the wrong value

    \end{Verbatim}

    \begin{Verbatim}[commandchars=\\\{\}]
{\color{incolor}In [{\color{incolor}11}]:} \PY{k}{try}\PY{p}{:}
             \PY{k}{raise} \PY{n+ne}{ValueError}\PY{p}{(}\PY{l+s}{\PYZsq{}}\PY{l+s}{This is the wrong value}\PY{l+s}{\PYZsq{}}\PY{p}{)}
         \PY{k}{except} \PY{n+ne}{ValueError}\PY{p}{:}
             \PY{k}{pass}
\end{Verbatim}

    \subsection{Importing and Manipulating
Data}\label{importing-and-manipulating-data}

Python includes operations for importing and exporting data from files
and binary objects, and third-party packages exist for database
connectivity. The easiest way to import data from a file is to parse
delimited text file, which can usually be exported from spreadsheets and
databases. In fact, file is a built-in type in python. Data may be read
from and written to regular files by specifying them as file objects:

    \begin{Verbatim}[commandchars=\\\{\}]
{\color{incolor}In [{\color{incolor}12}]:} \PY{n}{microbiome} \PY{o}{=} \PY{n+nb}{open}\PY{p}{(}\PY{l+s}{\PYZsq{}}\PY{l+s}{../data/microbiome.csv}\PY{l+s}{\PYZsq{}}\PY{p}{)}
\end{Verbatim}

    Here, a file containing microbiome data in a comma-delimited format is
opened, and assigned to an object, called \texttt{microbiome}. The next
step is to transfer the information in the file to a usable data
structure in Python. Since this dataset contains four variables, the
name of the taxon, the patient identifier (de-identified), the bacteria
count in tissue and the bacteria count in stool, it is convenient to use
a dictionary. This allows each variable to be specified by name.

First, a dictionary object is initialized, with appropriate keys and
corresponding lists, initially empty. Since the file has a header, we
can use it to generate an empty dict:

    \begin{Verbatim}[commandchars=\\\{\}]
{\color{incolor}In [{\color{incolor}13}]:} \PY{n}{column\PYZus{}names} \PY{o}{=} \PY{n}{microbiome}\PY{o}{.}\PY{n}{next}\PY{p}{(}\PY{p}{)}\PY{o}{.}\PY{n}{rstrip}\PY{p}{(}\PY{l+s}{\PYZsq{}}\PY{l+s+se}{\PYZbs{}n}\PY{l+s}{\PYZsq{}}\PY{p}{)}\PY{o}{.}\PY{n}{split}\PY{p}{(}\PY{l+s}{\PYZsq{}}\PY{l+s}{,}\PY{l+s}{\PYZsq{}}\PY{p}{)}
\end{Verbatim}

    \begin{Verbatim}[commandchars=\\\{\}]
{\color{incolor}In [{\color{incolor}14}]:} \PY{n}{mb\PYZus{}dict} \PY{o}{=} \PY{p}{\PYZob{}}\PY{n}{name}\PY{p}{:}\PY{p}{[}\PY{p}{]} \PY{k}{for} \PY{n}{name} \PY{o+ow}{in} \PY{n}{column\PYZus{}names}\PY{p}{\PYZcb{}}
\end{Verbatim}

    It is then a matter of looping over each line of the data. Python file
objects are iterable, essentially just a sequence of lines, and fit
naturally into a for statement.

    \begin{Verbatim}[commandchars=\\\{\}]
{\color{incolor}In [{\color{incolor}15}]:} \PY{k}{for} \PY{n}{line} \PY{o+ow}{in} \PY{n}{microbiome}\PY{p}{:}
             \PY{n}{taxon}\PY{p}{,}\PY{n}{patient}\PY{p}{,}\PY{n}{tissue}\PY{p}{,}\PY{n}{stool} \PY{o}{=} \PY{n}{line}\PY{o}{.}\PY{n}{rstrip}\PY{p}{(}\PY{l+s}{\PYZsq{}}\PY{l+s+se}{\PYZbs{}n}\PY{l+s}{\PYZsq{}}\PY{p}{)}\PY{o}{.}\PY{n}{split}\PY{p}{(}\PY{l+s}{\PYZsq{}}\PY{l+s}{,}\PY{l+s}{\PYZsq{}}\PY{p}{)}
             \PY{n}{mb\PYZus{}dict}\PY{p}{[}\PY{l+s}{\PYZsq{}}\PY{l+s}{Taxon}\PY{l+s}{\PYZsq{}}\PY{p}{]}\PY{o}{.}\PY{n}{append}\PY{p}{(}\PY{n}{taxon}\PY{p}{)}
             \PY{n}{mb\PYZus{}dict}\PY{p}{[}\PY{l+s}{\PYZsq{}}\PY{l+s}{Patient}\PY{l+s}{\PYZsq{}}\PY{p}{]}\PY{o}{.}\PY{n}{append}\PY{p}{(}\PY{n+nb}{int}\PY{p}{(}\PY{n}{patient}\PY{p}{)}\PY{p}{)}
             \PY{n}{mb\PYZus{}dict}\PY{p}{[}\PY{l+s}{\PYZsq{}}\PY{l+s}{Tissue}\PY{l+s}{\PYZsq{}}\PY{p}{]}\PY{o}{.}\PY{n}{append}\PY{p}{(}\PY{n+nb}{int}\PY{p}{(}\PY{n}{tissue}\PY{p}{)}\PY{p}{)}
             \PY{n}{mb\PYZus{}dict}\PY{p}{[}\PY{l+s}{\PYZsq{}}\PY{l+s}{Stool}\PY{l+s}{\PYZsq{}}\PY{p}{]}\PY{o}{.}\PY{n}{append}\PY{p}{(}\PY{n+nb}{int}\PY{p}{(}\PY{n}{stool}\PY{p}{)}\PY{p}{)}
\end{Verbatim}

    For each line in the file, data elements are split by the comma
delimiter, using the \texttt{split} method that is built-in to string
objects. Each datum is subsequently appended to the appropriate list
stored in the dictionary. After all the data is parsed, it is polite to
close the file:

    \begin{Verbatim}[commandchars=\\\{\}]
{\color{incolor}In [{\color{incolor}16}]:} \PY{n}{microbiome}\PY{o}{.}\PY{n}{close}\PY{p}{(}\PY{p}{)}
\end{Verbatim}

    The data can now be readily accessed by indexing the appropriate
variable by name:

    \begin{Verbatim}[commandchars=\\\{\}]
{\color{incolor}In [{\color{incolor}17}]:} \PY{n}{mb\PYZus{}dict}\PY{p}{[}\PY{l+s}{\PYZsq{}}\PY{l+s}{Tissue}\PY{l+s}{\PYZsq{}}\PY{p}{]}\PY{p}{[}\PY{p}{:}\PY{l+m+mi}{10}\PY{p}{]}
\end{Verbatim}

            \begin{Verbatim}[commandchars=\\\{\}]
{\color{outcolor}Out[{\color{outcolor}17}]:} [632, 136, 1174, 408, 831, 693, 718, 173, 228, 162]
\end{Verbatim}
        
    A second approach to importing data involves interfacing directly with a
relational database management system. Relational databases are far more
efficient for storing, maintaining and querying data than plain text
files or spreadsheets, particularly for large datasets or multiple
tables. A number of third parties have created packages for database
access in Python. For example, \texttt{sqlite3} is a package that
provides connectivity for SQLite databases:

    \begin{Verbatim}[commandchars=\\\{\}]
{\color{incolor}In [{\color{incolor}18}]:} \PY{k+kn}{import} \PY{n+nn}{sqlite3}		\PY{c}{\PYZsh{} import database package, and connect}
         \PY{n}{db} \PY{o}{=} \PY{n}{sqlite3}\PY{o}{.}\PY{n}{connect}\PY{p}{(}\PY{n}{database}\PY{o}{=}\PY{l+s}{\PYZsq{}}\PY{l+s}{../data/baseball\PYZhy{}archive\PYZhy{}2011.sqlite}\PY{l+s}{\PYZsq{}}\PY{p}{)}
         \PY{n}{cur} \PY{o}{=} \PY{n}{db}\PY{o}{.}\PY{n}{cursor}\PY{p}{(}\PY{p}{)}	\PY{c}{\PYZsh{} create a cursor object to mediate}
         				    \PY{c}{\PYZsh{} communication with database}
\end{Verbatim}

    \begin{Verbatim}[commandchars=\\\{\}]
{\color{incolor}In [{\color{incolor}19}]:} \PY{c}{\PYZsh{} run query}
         \PY{n}{cur}\PY{o}{.}\PY{n}{execute}\PY{p}{(}\PY{l+s}{\PYZsq{}}\PY{l+s}{SELECT playerid, HR, SB FROM Batting WHERE yearID=1970}\PY{l+s}{\PYZsq{}}\PY{p}{)}
         \PY{n}{baseball} \PY{o}{=} \PY{n}{cur}\PY{o}{.}\PY{n}{fetchall}\PY{p}{(}\PY{p}{)}	\PY{c}{\PYZsh{} fetch data, and assign to variable}
         \PY{n}{baseball}\PY{p}{[}\PY{p}{:}\PY{l+m+mi}{10}\PY{p}{]}
\end{Verbatim}

            \begin{Verbatim}[commandchars=\\\{\}]
{\color{outcolor}Out[{\color{outcolor}19}]:} [(u'aaronha01', 38, 9),
          (u'aaronto01', 2, 0),
          (u'abernte02', 0, 0),
          (u'abernte02', 0, 0),
          (u'abernte02', 0, 0),
          (u'acosted01', 0, 0),
          (u'adairje01', 0, 0),
          (u'ageeto01', 24, 31),
          (u'aguirha01', 0, 0),
          (u'akerja01', 0, 0)]
\end{Verbatim}
        
    \subsection{Functions}\label{functions}

Python uses the \texttt{def} statement to encapsulate code into a
callable function. Here again is a very simple Python function:

    \begin{Verbatim}[commandchars=\\\{\}]
{\color{incolor}In [{\color{incolor}20}]:} \PY{c}{\PYZsh{} Function for calulating the mean of some data}
         \PY{k}{def} \PY{n+nf}{mean}\PY{p}{(}\PY{n}{data}\PY{p}{)}\PY{p}{:}
         
             \PY{c}{\PYZsh{} Initialize sum to zero}
             \PY{n}{sum\PYZus{}x} \PY{o}{=} \PY{l+m+mf}{0.0}
         
             \PY{c}{\PYZsh{} Loop over data}
             \PY{k}{for} \PY{n}{x} \PY{o+ow}{in} \PY{n}{data}\PY{p}{:}
         
                 \PY{c}{\PYZsh{} Add to sum}
                 \PY{n}{sum\PYZus{}x} \PY{o}{+}\PY{o}{=} \PY{n}{x} 
             
             \PY{c}{\PYZsh{} Divide by number of elements in list, and return}
             \PY{k}{return} \PY{n}{sum\PYZus{}x} \PY{o}{/} \PY{n+nb}{len}\PY{p}{(}\PY{n}{data}\PY{p}{)}
\end{Verbatim}

    As we can see, arguments are specified in parentheses following the
function name. If there are sensible ``default'' values, they can be
specified as a \emph{keyword argument}.

    \begin{Verbatim}[commandchars=\\\{\}]
{\color{incolor}In [{\color{incolor}21}]:} \PY{k}{def} \PY{n+nf}{var}\PY{p}{(}\PY{n}{data}\PY{p}{,} \PY{n}{sample}\PY{o}{=}\PY{n+nb+bp}{True}\PY{p}{)}\PY{p}{:}
         
             \PY{c}{\PYZsh{} Get mean of data from function above}
             \PY{n}{x\PYZus{}bar} \PY{o}{=} \PY{n}{mean}\PY{p}{(}\PY{n}{data}\PY{p}{)}
         
             \PY{c}{\PYZsh{} Do sum of squares in one line}
             \PY{n}{sum\PYZus{}squares} \PY{o}{=} \PY{n+nb}{sum}\PY{p}{(}\PY{p}{[}\PY{p}{(}\PY{n}{x} \PY{o}{\PYZhy{}} \PY{n}{x\PYZus{}bar}\PY{p}{)}\PY{o}{*}\PY{o}{*}\PY{l+m+mi}{2} \PY{k}{for} \PY{n}{x} \PY{o+ow}{in} \PY{n}{data}\PY{p}{]}\PY{p}{)}
         
             \PY{c}{\PYZsh{} Divide by n\PYZhy{}1 and return}
             \PY{k}{if} \PY{n}{sample}\PY{p}{:}
                 \PY{k}{return} \PY{n}{sum\PYZus{}squares}\PY{o}{/}\PY{p}{(}\PY{n+nb}{len}\PY{p}{(}\PY{n}{data}\PY{p}{)}\PY{o}{\PYZhy{}}\PY{l+m+mi}{1}\PY{p}{)}
             \PY{k}{return} \PY{n}{sum\PYZus{}squares}\PY{o}{/}\PY{n+nb}{len}\PY{p}{(}\PY{n}{data}\PY{p}{)}
\end{Verbatim}

    Non-keyword arguments must always predede keyword arguments, and must
always be presented in order; order is not important for keyword
arguments.

Arguments can also be passed to functions as a
\texttt{tuple}/\texttt{list}/\texttt{dict} using the asterisk notation.

    \begin{Verbatim}[commandchars=\\\{\}]
{\color{incolor}In [{\color{incolor}22}]:} \PY{k}{def} \PY{n+nf}{some\PYZus{}computation}\PY{p}{(}\PY{n}{a}\PY{o}{=}\PY{o}{\PYZhy{}}\PY{l+m+mi}{1}\PY{p}{,} \PY{n}{b}\PY{o}{=}\PY{l+m+mf}{4.3}\PY{p}{,} \PY{n}{c}\PY{o}{=}\PY{l+m+mi}{7}\PY{p}{)}\PY{p}{:}
             \PY{k}{return} \PY{p}{(}\PY{n}{a} \PY{o}{+} \PY{n}{b}\PY{p}{)} \PY{o}{/} \PY{n+nb}{float}\PY{p}{(}\PY{n}{c}\PY{p}{)}
         
         \PY{n}{args} \PY{o}{=} \PY{p}{(}\PY{l+m+mi}{5}\PY{p}{,} \PY{l+m+mi}{4}\PY{p}{,} \PY{l+m+mi}{3}\PY{p}{)}
         \PY{n}{some\PYZus{}computation}\PY{p}{(}\PY{o}{*}\PY{n}{args}\PY{p}{)}
\end{Verbatim}

            \begin{Verbatim}[commandchars=\\\{\}]
{\color{outcolor}Out[{\color{outcolor}22}]:} 3.0
\end{Verbatim}
        
    \begin{Verbatim}[commandchars=\\\{\}]
{\color{incolor}In [{\color{incolor}23}]:} \PY{n}{kwargs} \PY{o}{=} \PY{p}{\PYZob{}}\PY{l+s}{\PYZsq{}}\PY{l+s}{b}\PY{l+s}{\PYZsq{}}\PY{p}{:}\PY{l+m+mi}{4}\PY{p}{,} \PY{l+s}{\PYZsq{}}\PY{l+s}{a}\PY{l+s}{\PYZsq{}}\PY{p}{:}\PY{l+m+mi}{5}\PY{p}{,} \PY{l+s}{\PYZsq{}}\PY{l+s}{c}\PY{l+s}{\PYZsq{}}\PY{p}{:}\PY{l+m+mi}{3}\PY{p}{\PYZcb{}}
         \PY{n}{some\PYZus{}computation}\PY{p}{(}\PY{o}{*}\PY{o}{*}\PY{n}{kwargs}\PY{p}{)}
\end{Verbatim}

            \begin{Verbatim}[commandchars=\\\{\}]
{\color{outcolor}Out[{\color{outcolor}23}]:} 3.0
\end{Verbatim}
        
    The \texttt{lambda} statement creates anonymous one-line functions that
can simply be assigned to a name.

    \begin{Verbatim}[commandchars=\\\{\}]
{\color{incolor}In [{\color{incolor}24}]:} \PY{k+kn}{import} \PY{n+nn}{numpy} \PY{k+kn}{as} \PY{n+nn}{np}
         \PY{n}{normalize} \PY{o}{=} \PY{k}{lambda} \PY{n}{data}\PY{p}{:} \PY{p}{(}\PY{n}{np}\PY{o}{.}\PY{n}{array}\PY{p}{(}\PY{n}{data}\PY{p}{)} \PY{o}{\PYZhy{}} \PY{n}{np}\PY{o}{.}\PY{n}{mean}\PY{p}{(}\PY{n}{data}\PY{p}{)}\PY{p}{)} \PY{o}{/} \PY{n}{np}\PY{o}{.}\PY{n}{std}\PY{p}{(}\PY{n}{data}\PY{p}{)}
\end{Verbatim}

    or not:

    \begin{Verbatim}[commandchars=\\\{\}]
{\color{incolor}In [{\color{incolor}25}]:} \PY{p}{(}\PY{k}{lambda} \PY{n}{data}\PY{p}{:} \PY{p}{(}\PY{n}{np}\PY{o}{.}\PY{n}{array}\PY{p}{(}\PY{n}{data}\PY{p}{)} \PY{o}{\PYZhy{}} \PY{n}{np}\PY{o}{.}\PY{n}{mean}\PY{p}{(}\PY{n}{data}\PY{p}{)}\PY{p}{)} \PY{o}{/} \PY{n}{np}\PY{o}{.}\PY{n}{std}\PY{p}{(}\PY{n}{data}\PY{p}{)}\PY{p}{)}\PY{p}{(}\PY{p}{[}\PY{l+m+mi}{5}\PY{p}{,}\PY{l+m+mi}{8}\PY{p}{,}\PY{l+m+mi}{3}\PY{p}{,}\PY{l+m+mi}{8}\PY{p}{,}\PY{l+m+mi}{3}\PY{p}{,}\PY{l+m+mi}{1}\PY{p}{,}\PY{l+m+mi}{2}\PY{p}{,}\PY{l+m+mi}{1}\PY{p}{]}\PY{p}{)}
\end{Verbatim}

            \begin{Verbatim}[commandchars=\\\{\}]
{\color{outcolor}Out[{\color{outcolor}25}]:} array([ 0.42192651,  1.54706386, -0.32816506,  1.54706386, -0.32816506,
                -1.07825663, -0.70321085, -1.07825663])
\end{Verbatim}
        
    Python has several built-in, higher-order functions that are useful.

    \begin{Verbatim}[commandchars=\\\{\}]
{\color{incolor}In [{\color{incolor}26}]:} \PY{n+nb}{filter}\PY{p}{(}\PY{k}{lambda} \PY{n}{x}\PY{p}{:} \PY{n}{x} \PY{o}{\PYZgt{}} \PY{l+m+mi}{5}\PY{p}{,} \PY{n+nb}{range}\PY{p}{(}\PY{l+m+mi}{10}\PY{p}{)}\PY{p}{)}
\end{Verbatim}

            \begin{Verbatim}[commandchars=\\\{\}]
{\color{outcolor}Out[{\color{outcolor}26}]:} [6, 7, 8, 9]
\end{Verbatim}
        
    \begin{Verbatim}[commandchars=\\\{\}]
{\color{incolor}In [{\color{incolor}27}]:} \PY{n+nb}{abs}\PY{p}{(}\PY{p}{[}\PY{l+m+mi}{5}\PY{p}{,}\PY{o}{\PYZhy{}}\PY{l+m+mi}{6}\PY{p}{]}\PY{p}{)}
\end{Verbatim}

    \begin{Verbatim}[commandchars=\\\{\}]

        ---------------------------------------------------------------------------
    TypeError                                 Traceback (most recent call last)

        <ipython-input-27-2fc5ae37d9b8> in <module>()
    ----> 1 abs([5,-6])
    

        TypeError: bad operand type for abs(): 'list'

    \end{Verbatim}

    \begin{Verbatim}[commandchars=\\\{\}]
{\color{incolor}In [{\color{incolor}28}]:} \PY{n+nb}{map}\PY{p}{(}\PY{n+nb}{abs}\PY{p}{,} \PY{p}{[}\PY{l+m+mi}{5}\PY{p}{,} \PY{o}{\PYZhy{}}\PY{l+m+mi}{6}\PY{p}{]}\PY{p}{)}
\end{Verbatim}

            \begin{Verbatim}[commandchars=\\\{\}]
{\color{outcolor}Out[{\color{outcolor}28}]:} [5, 6]
\end{Verbatim}
        
    \subsection{Example: Least Squares
Estimation}\label{example-least-squares-estimation}

Lets try coding a statistical function. Suppose we want to estimate the
parameters of a simple linear regression model. The objective of
regression analysis is to specify an equation that will predict some
response variable $Y$ based on a set of predictor variables $X$. This is
done by fitting parameter values $\beta$ of a regression model using
extant data for $X$ and $Y$. This equation has the form:

\[Y = X\beta + \epsilon\]

where $\epsilon$ is a vector of errors. One way to fit this model is
using the method of \emph{least squares}, which is given by:

\[\hat{\beta} = (X\prime X)^{-1}X\prime Y\]

We can write a function that calculates this estimate, with the help of
some functions from other modules:

    \begin{Verbatim}[commandchars=\\\{\}]
{\color{incolor}In [{\color{incolor}29}]:} \PY{k+kn}{from} \PY{n+nn}{numpy.linalg} \PY{k+kn}{import} \PY{n}{inv}
         \PY{k+kn}{from} \PY{n+nn}{numpy} \PY{k+kn}{import} \PY{n}{transpose}\PY{p}{,} \PY{n}{array}\PY{p}{,} \PY{n}{dot}
\end{Verbatim}

    We will call this function \texttt{solve}, requiring the predictor and
response variables as arguments. For simplicity, we will restrict the
function to univariate regression, whereby only a single slope and
intercept are estimated:

    \begin{Verbatim}[commandchars=\\\{\}]
{\color{incolor}In [{\color{incolor}30}]:} \PY{k}{def} \PY{n+nf}{solve}\PY{p}{(}\PY{n}{x}\PY{p}{,}\PY{n}{y}\PY{p}{)}\PY{p}{:}
             \PY{l+s}{\PYZsq{}}\PY{l+s}{Estimates regession coefficents from data}\PY{l+s}{\PYZsq{}}
         
             \PY{l+s+sd}{\PYZsq{}\PYZsq{}\PYZsq{}}
         \PY{l+s+sd}{    The first step is to specify the design matrix. For this, }
         \PY{l+s+sd}{    we need to create a vector of ones (corresponding to the intercept term, }
         \PY{l+s+sd}{    and along with x, create a n x 2 array:}
         \PY{l+s+sd}{    \PYZsq{}\PYZsq{}\PYZsq{}}
             \PY{n}{X} \PY{o}{=} \PY{n}{array}\PY{p}{(}\PY{p}{[}\PY{p}{[}\PY{l+m+mi}{1}\PY{p}{]}\PY{o}{*}\PY{n+nb}{len}\PY{p}{(}\PY{n}{x}\PY{p}{)}\PY{p}{,} \PY{n}{x}\PY{p}{]}\PY{p}{)}
         
             \PY{l+s+sd}{\PYZsq{}\PYZsq{}\PYZsq{}}
         \PY{l+s+sd}{    An array is a data structure from the numpy package, similar to a list, }
         \PY{l+s+sd}{    but allowing for multiple dimensions. Next, we calculate the transpose of x, }
         \PY{l+s+sd}{    using another numpy function, transpose}
         \PY{l+s+sd}{    \PYZsq{}\PYZsq{}\PYZsq{}}
             \PY{n}{Xt} \PY{o}{=} \PY{n}{transpose}\PY{p}{(}\PY{n}{X}\PY{p}{)}
         
             \PY{l+s+sd}{\PYZsq{}\PYZsq{}\PYZsq{}}
         \PY{l+s+sd}{    Finally, we use the matrix multiplication function dot, also from numpy }
         \PY{l+s+sd}{    to calculate the dot product. The inverse function is provided by the LinearAlgebra }
         \PY{l+s+sd}{    package. Provided that x is not singular (which would raise an exception), this }
         \PY{l+s+sd}{    yields estimates of the intercept and slope, as an array}
         \PY{l+s+sd}{    \PYZsq{}\PYZsq{}\PYZsq{}}
             \PY{n}{b\PYZus{}hat} \PY{o}{=} \PY{n}{dot}\PY{p}{(}\PY{n}{inv}\PY{p}{(}\PY{n}{dot}\PY{p}{(}\PY{n}{X}\PY{p}{,}\PY{n}{Xt}\PY{p}{)}\PY{p}{)}\PY{p}{,} \PY{n}{dot}\PY{p}{(}\PY{n}{X}\PY{p}{,}\PY{n}{y}\PY{p}{)}\PY{p}{)}
         
             \PY{k}{return} \PY{n}{b\PYZus{}hat} 
\end{Verbatim}

    Here is solve in action:

    \begin{Verbatim}[commandchars=\\\{\}]
{\color{incolor}In [{\color{incolor}31}]:} \PY{n}{solve}\PY{p}{(}\PY{p}{(}\PY{l+m+mi}{10}\PY{p}{,}\PY{l+m+mi}{5}\PY{p}{,}\PY{l+m+mi}{10}\PY{p}{,}\PY{l+m+mi}{11}\PY{p}{,}\PY{l+m+mi}{14}\PY{p}{)}\PY{p}{,}\PY{p}{(}\PY{o}{\PYZhy{}}\PY{l+m+mi}{4}\PY{p}{,}\PY{l+m+mi}{3}\PY{p}{,}\PY{l+m+mi}{0}\PY{p}{,}\PY{l+m+mi}{23}\PY{p}{,}\PY{l+m+mf}{0.6}\PY{p}{)}\PY{p}{)}
\end{Verbatim}

            \begin{Verbatim}[commandchars=\\\{\}]
{\color{outcolor}Out[{\color{outcolor}31}]:} array([ 2.04380952,  0.24761905])
\end{Verbatim}
        
    \subsection{Object-oriented
Programming}\label{object-oriented-programming}

As previously stated, Python is an object-oriented programming (OOP)
language, in contrast to procedural languages like FORTRAN and C. As the
name implies, object-oriented languages employ objects to create
convenient abstractions of data structures. This allows for more
flexible programs, fewer lines of code, and a more natural programming
paradigm in general. An object is simply a modular unit of data and
associated functions, related to the state and behavior, respectively,
of some abstract entity. Object-oriented languages group similar objects
into classes. For example, consider a Python class representing a bird:

    \begin{Verbatim}[commandchars=\\\{\}]
{\color{incolor}In [{\color{incolor}32}]:} \PY{k}{class} \PY{n+nc}{bird}\PY{p}{:}
             \PY{c}{\PYZsh{} Class representing a bird}
         
             \PY{n}{name} \PY{o}{=} \PY{l+s}{\PYZsq{}}\PY{l+s}{bird}\PY{l+s}{\PYZsq{}}
             
             \PY{k}{def} \PY{n+nf}{\PYZus{}\PYZus{}init\PYZus{}\PYZus{}}\PY{p}{(}\PY{n+nb+bp}{self}\PY{p}{,} \PY{n}{sex}\PY{p}{)}\PY{p}{:}
                 \PY{c}{\PYZsh{} Initialization method}
                 
                 \PY{n+nb+bp}{self}\PY{o}{.}\PY{n}{sex} \PY{o}{=} \PY{n}{sex}
         	
             \PY{k}{def} \PY{n+nf}{fly}\PY{p}{(}\PY{n+nb+bp}{self}\PY{p}{)}\PY{p}{:}
                 \PY{c}{\PYZsh{} Makes bird fly}
         
                 \PY{k}{print}\PY{p}{(}\PY{l+s}{\PYZsq{}}\PY{l+s}{Flying!}\PY{l+s}{\PYZsq{}}\PY{p}{)}
                 
             \PY{k}{def} \PY{n+nf}{nest}\PY{p}{(}\PY{n+nb+bp}{self}\PY{p}{)}\PY{p}{:}
                 \PY{c}{\PYZsh{} Makes bird build nest}
         
                 \PY{k}{print}\PY{p}{(}\PY{l+s}{\PYZsq{}}\PY{l+s}{Building nest ...}\PY{l+s}{\PYZsq{}}\PY{p}{)}
                 
             \PY{n+nd}{@classmethod}
             \PY{k}{def} \PY{n+nf}{get\PYZus{}name}\PY{p}{(}\PY{n}{cls}\PY{p}{)}\PY{p}{:}
                 \PY{c}{\PYZsh{} Class methods are shared among instances}
                 
                 \PY{k}{return} \PY{n}{cls}\PY{o}{.}\PY{n}{name}
\end{Verbatim}

    You will notice that this \texttt{bird} class is simply a container for
two functions (called \emph{methods} in Python), \texttt{fly} and
\texttt{nest}, as well as one attribute, \texttt{name}. The methods
represent functions in common with all members of this class. You can
run this code in Python, and create birds:

    \begin{Verbatim}[commandchars=\\\{\}]
{\color{incolor}In [{\color{incolor}33}]:} \PY{n}{Tweety} \PY{o}{=} \PY{n}{bird}\PY{p}{(}\PY{l+s}{\PYZsq{}}\PY{l+s}{male}\PY{l+s}{\PYZsq{}}\PY{p}{)}
         \PY{n}{Tweety}\PY{o}{.}\PY{n}{name}
\end{Verbatim}

            \begin{Verbatim}[commandchars=\\\{\}]
{\color{outcolor}Out[{\color{outcolor}33}]:} 'bird'
\end{Verbatim}
        
    \begin{Verbatim}[commandchars=\\\{\}]
{\color{incolor}In [{\color{incolor}34}]:} \PY{n}{Tweety}\PY{o}{.}\PY{n}{fly}\PY{p}{(}\PY{p}{)}
\end{Verbatim}

    \begin{Verbatim}[commandchars=\\\{\}]
Flying!
    \end{Verbatim}

    \begin{Verbatim}[commandchars=\\\{\}]
{\color{incolor}In [{\color{incolor}35}]:} \PY{n}{Foghorn} \PY{o}{=} \PY{n}{bird}\PY{p}{(}\PY{l+s}{\PYZsq{}}\PY{l+s}{male}\PY{l+s}{\PYZsq{}}\PY{p}{)}
         \PY{n}{Foghorn}\PY{o}{.}\PY{n}{nest}\PY{p}{(}\PY{p}{)}
\end{Verbatim}

    \begin{Verbatim}[commandchars=\\\{\}]
Building nest \ldots
    \end{Verbatim}

    A \texttt{classmethod} can be called without instantiating an object.

    \begin{Verbatim}[commandchars=\\\{\}]
{\color{incolor}In [{\color{incolor}36}]:} \PY{n}{bird}\PY{o}{.}\PY{n}{get\PYZus{}name}\PY{p}{(}\PY{p}{)}
\end{Verbatim}

            \begin{Verbatim}[commandchars=\\\{\}]
{\color{outcolor}Out[{\color{outcolor}36}]:} 'bird'
\end{Verbatim}
        
    As many instances of the \texttt{bird} class can be generated as
desired, though it may quickly become boring. One of the important
benefits of using object-oriented classes is code re-use. For example,
we may want more specific kinds of birds, with unique functionality:

    \begin{Verbatim}[commandchars=\\\{\}]
{\color{incolor}In [{\color{incolor}37}]:} \PY{k}{class} \PY{n+nc}{duck}\PY{p}{(}\PY{n}{bird}\PY{p}{)}\PY{p}{:}
             \PY{c}{\PYZsh{} Duck is a subclass of bird}
         
             \PY{n}{name} \PY{o}{=} \PY{l+s}{\PYZsq{}}\PY{l+s}{duck}\PY{l+s}{\PYZsq{}}
             
             \PY{k}{def} \PY{n+nf}{swim}\PY{p}{(}\PY{n+nb+bp}{self}\PY{p}{)}\PY{p}{:}
                 \PY{c}{\PYZsh{} Ducks can swim}
         
                 \PY{k}{print}\PY{p}{(}\PY{l+s}{\PYZsq{}}\PY{l+s}{Swimming!}\PY{l+s}{\PYZsq{}}\PY{p}{)}
         	
             \PY{k}{def} \PY{n+nf}{quack}\PY{p}{(}\PY{n+nb+bp}{self}\PY{p}{,}\PY{n}{n}\PY{p}{)}\PY{p}{:}
                 \PY{c}{\PYZsh{} Ducks can quack}
             
                 \PY{k}{print}\PY{p}{(}\PY{l+s}{\PYZsq{}}\PY{l+s}{Quack! }\PY{l+s}{\PYZsq{}} \PY{o}{*} \PY{n}{n}\PY{p}{)}
\end{Verbatim}

    Notice that this new \texttt{duck} class refers to the \texttt{bird}
class in parentheses after the class declaration; this is called
\textbf{inheritance}. The subclass \texttt{duck} automatically inherits
all of the variables and methods of the superclass, but allows new
functions or variables to be added. In addition to flying and
best-building, our duck can also swim and quack:

    \begin{Verbatim}[commandchars=\\\{\}]
{\color{incolor}In [{\color{incolor}38}]:} \PY{n}{Daffy} \PY{o}{=} \PY{n}{duck}\PY{p}{(}\PY{l+s}{\PYZsq{}}\PY{l+s}{male}\PY{l+s}{\PYZsq{}}\PY{p}{)}
         \PY{n}{Daffy}\PY{o}{.}\PY{n}{swim}\PY{p}{(}\PY{p}{)}
\end{Verbatim}

    \begin{Verbatim}[commandchars=\\\{\}]
Swimming!
    \end{Verbatim}

    \begin{Verbatim}[commandchars=\\\{\}]
{\color{incolor}In [{\color{incolor}39}]:} \PY{n}{Daffy}\PY{o}{.}\PY{n}{quack}\PY{p}{(}\PY{l+m+mi}{3}\PY{p}{)}
\end{Verbatim}

    \begin{Verbatim}[commandchars=\\\{\}]
Quack! Quack! Quack!
    \end{Verbatim}

    \begin{Verbatim}[commandchars=\\\{\}]
{\color{incolor}In [{\color{incolor}40}]:} \PY{n}{Daffy}\PY{o}{.}\PY{n}{nest}\PY{p}{(}\PY{p}{)}
\end{Verbatim}

    \begin{Verbatim}[commandchars=\\\{\}]
Building nest \ldots
    \end{Verbatim}

    Along with adding new variables and methods, a subclass can also
override existing variables and methods of the superclass. For example,
one might define \texttt{fly} in the \texttt{duck} subclass to return an
entirely different string. It is easy to see how inheritance promotes
code re-use, sometimes dramatically reducing development time. Classes
which are very similar need not be coded repetitiously, but rather, just
extended from a single superclass.

    This brief introduction to object-oriented programming is intended only
to introduce new users of Python to this programming paradigm. There are
many more salient object-oriented topics, including interfaces,
composition, and introspection. I encourage interested readers to refer
to any number of current Python and OOP books for a more comprehensive
treatment.

    \subsection{In Python, everything is an
object}\label{in-python-everything-is-an-object}

Everything (and I mean \emph{everything}) in Python is an object, in the
sense that they possess attributes, such as methods and variables, that
we usually associate with more ``structured'' objects like those we
created above.

Check it out:

    \begin{Verbatim}[commandchars=\\\{\}]
{\color{incolor}In [{\color{incolor}41}]:} \PY{n+nb}{dir}\PY{p}{(}\PY{l+m+mi}{1}\PY{p}{)}
\end{Verbatim}

            \begin{Verbatim}[commandchars=\\\{\}]
{\color{outcolor}Out[{\color{outcolor}41}]:} ['\_\_abs\_\_',
          '\_\_add\_\_',
          '\_\_and\_\_',
          '\_\_class\_\_',
          '\_\_cmp\_\_',
          '\_\_coerce\_\_',
          '\_\_delattr\_\_',
          '\_\_div\_\_',
          '\_\_divmod\_\_',
          '\_\_doc\_\_',
          '\_\_float\_\_',
          '\_\_floordiv\_\_',
          '\_\_format\_\_',
          '\_\_getattribute\_\_',
          '\_\_getnewargs\_\_',
          '\_\_hash\_\_',
          '\_\_hex\_\_',
          '\_\_index\_\_',
          '\_\_init\_\_',
          '\_\_int\_\_',
          '\_\_invert\_\_',
          '\_\_long\_\_',
          '\_\_lshift\_\_',
          '\_\_mod\_\_',
          '\_\_mul\_\_',
          '\_\_neg\_\_',
          '\_\_new\_\_',
          '\_\_nonzero\_\_',
          '\_\_oct\_\_',
          '\_\_or\_\_',
          '\_\_pos\_\_',
          '\_\_pow\_\_',
          '\_\_radd\_\_',
          '\_\_rand\_\_',
          '\_\_rdiv\_\_',
          '\_\_rdivmod\_\_',
          '\_\_reduce\_\_',
          '\_\_reduce\_ex\_\_',
          '\_\_repr\_\_',
          '\_\_rfloordiv\_\_',
          '\_\_rlshift\_\_',
          '\_\_rmod\_\_',
          '\_\_rmul\_\_',
          '\_\_ror\_\_',
          '\_\_rpow\_\_',
          '\_\_rrshift\_\_',
          '\_\_rshift\_\_',
          '\_\_rsub\_\_',
          '\_\_rtruediv\_\_',
          '\_\_rxor\_\_',
          '\_\_setattr\_\_',
          '\_\_sizeof\_\_',
          '\_\_str\_\_',
          '\_\_sub\_\_',
          '\_\_subclasshook\_\_',
          '\_\_truediv\_\_',
          '\_\_trunc\_\_',
          '\_\_xor\_\_',
          'bit\_length',
          'conjugate',
          'denominator',
          'imag',
          'numerator',
          'real']
\end{Verbatim}
        
    \begin{Verbatim}[commandchars=\\\{\}]
{\color{incolor}In [{\color{incolor}42}]:} \PY{p}{(}\PY{l+m+mi}{1}\PY{p}{)}\PY{o}{.}\PY{n}{bit\PYZus{}length}\PY{p}{(}\PY{p}{)}
\end{Verbatim}

            \begin{Verbatim}[commandchars=\\\{\}]
{\color{outcolor}Out[{\color{outcolor}42}]:} 1
\end{Verbatim}
        
    This has implications for how assignment works in Python.

Let's create a trivial class:

    \begin{Verbatim}[commandchars=\\\{\}]
{\color{incolor}In [{\color{incolor}43}]:} \PY{k}{class} \PY{n+nc}{Thing}\PY{p}{:} \PY{k}{pass}
\end{Verbatim}

    and instantiate it:

    \begin{Verbatim}[commandchars=\\\{\}]
{\color{incolor}In [{\color{incolor}44}]:} \PY{n}{x} \PY{o}{=} \PY{n}{Thing}\PY{p}{(}\PY{p}{)}
         \PY{n}{x}
\end{Verbatim}

            \begin{Verbatim}[commandchars=\\\{\}]
{\color{outcolor}Out[{\color{outcolor}44}]:} <\_\_main\_\_.Thing instance at 0x10b88c1b8>
\end{Verbatim}
        
    Here, \texttt{x} is simply a ``label'' for the object that we created
when calling \texttt{Thing}. That object resides at the memory location
that is identified when we print \texttt{x}. Notice that if we create
another \texttt{Thing}, we create an new object, and give it a label. We
know it is a new object because it has its own memory location.

    \begin{Verbatim}[commandchars=\\\{\}]
{\color{incolor}In [{\color{incolor}45}]:} \PY{n}{y} \PY{o}{=} \PY{n}{Thing}\PY{p}{(}\PY{p}{)}
         \PY{n}{y}
\end{Verbatim}

            \begin{Verbatim}[commandchars=\\\{\}]
{\color{outcolor}Out[{\color{outcolor}45}]:} <\_\_main\_\_.Thing instance at 0x10b88c0e0>
\end{Verbatim}
        
    What if we assign \texttt{x} to \texttt{z}?

    \begin{Verbatim}[commandchars=\\\{\}]
{\color{incolor}In [{\color{incolor}46}]:} \PY{n}{z} \PY{o}{=} \PY{n}{x}
         \PY{n}{z}
\end{Verbatim}

            \begin{Verbatim}[commandchars=\\\{\}]
{\color{outcolor}Out[{\color{outcolor}46}]:} <\_\_main\_\_.Thing instance at 0x10b88c1b8>
\end{Verbatim}
        
    We see that the object labeled with \texttt{z} is the same as the object
as that labeled with \texttt{x}. So, we say that \texttt{z} is a label
(or name) with a \emph{binding} to the object created by \texttt{Thing}.

So, there are no ``variables'', in the sense of a container for values,
in Python. There are only labels and bindings.

    \begin{Verbatim}[commandchars=\\\{\}]
{\color{incolor}In [{\color{incolor}47}]:} \PY{n}{x}\PY{o}{.}\PY{n}{name} \PY{o}{=} \PY{l+s}{\PYZsq{}}\PY{l+s}{thing x}\PY{l+s}{\PYZsq{}}
\end{Verbatim}

    \begin{Verbatim}[commandchars=\\\{\}]
{\color{incolor}In [{\color{incolor}48}]:} \PY{n}{z}\PY{o}{.}\PY{n}{name}
\end{Verbatim}

            \begin{Verbatim}[commandchars=\\\{\}]
{\color{outcolor}Out[{\color{outcolor}48}]:} 'thing x'
\end{Verbatim}
        
    This can get you into trouble. Consider the following (seemingly
inoccuous) way of creating a dictionary of emtpy lists:

    \begin{Verbatim}[commandchars=\\\{\}]
{\color{incolor}In [{\color{incolor}49}]:} \PY{n}{evil\PYZus{}dict} \PY{o}{=} \PY{n+nb}{dict}\PY{o}{.}\PY{n}{fromkeys}\PY{p}{(}\PY{n}{column\PYZus{}names}\PY{p}{,} \PY{p}{[}\PY{p}{]}\PY{p}{)}
         \PY{n}{evil\PYZus{}dict}
\end{Verbatim}

            \begin{Verbatim}[commandchars=\\\{\}]
{\color{outcolor}Out[{\color{outcolor}49}]:} \{'Patient': [], 'Stool': [], 'Taxon': [], 'Tissue': []\}
\end{Verbatim}
        
    \begin{Verbatim}[commandchars=\\\{\}]
{\color{incolor}In [{\color{incolor}50}]:} \PY{n}{evil\PYZus{}dict}\PY{p}{[}\PY{l+s}{\PYZsq{}}\PY{l+s}{Tissue}\PY{l+s}{\PYZsq{}}\PY{p}{]}\PY{o}{.}\PY{n}{append}\PY{p}{(}\PY{l+m+mi}{5}\PY{p}{)}
\end{Verbatim}

    \begin{Verbatim}[commandchars=\\\{\}]
{\color{incolor}In [{\color{incolor}51}]:} \PY{n}{evil\PYZus{}dict}
\end{Verbatim}

            \begin{Verbatim}[commandchars=\\\{\}]
{\color{outcolor}Out[{\color{outcolor}51}]:} \{'Patient': [5], 'Stool': [5], 'Taxon': [5], 'Tissue': [5]\}
\end{Verbatim}
        
    Why did this happen?

    \subsection{Generators}\label{generators}

When a Python functions is called, it creates a namespace for the
function, executes the code that comprises the function (creating
objects inside the namespace as required), and returns some result to
its caller. After the return, everything inside the namespace (including
the namespace itself) is gone, and is created anew when the function is
called again.

However, one particular class of functions in Python breaks this
pattern, returning a value to the caller while still active, and able to
return subsequent values as needed. Python \textbf{\emph{generators}}
employ \texttt{yield} statements in place of \texttt{return}, allowing a
sequence of values to be generated without having to create a new
function namespace each time. In other languages, this construct is
known as a \emph{coroutine}.

For example, we may want to have a function that returns a sequence of
values; let's consider, for a simple illustration, the Fibonacci
sequence:

\[F_i = F_{i-2} + F_{i-1}\]

its certaintly possible to write a standard Python function that returns
however many Fibonacci numbers that we need:

    \begin{Verbatim}[commandchars=\\\{\}]
{\color{incolor}In [{\color{incolor}52}]:} \PY{k}{def} \PY{n+nf}{fibonacci}\PY{p}{(}\PY{n}{size}\PY{p}{)}\PY{p}{:}
             \PY{n}{F} \PY{o}{=} \PY{n}{np}\PY{o}{.}\PY{n}{empty}\PY{p}{(}\PY{n}{size}\PY{p}{,} \PY{l+s}{\PYZsq{}}\PY{l+s}{int}\PY{l+s}{\PYZsq{}}\PY{p}{)}
             \PY{n}{a}\PY{p}{,} \PY{n}{b} \PY{o}{=} \PY{l+m+mi}{0}\PY{p}{,} \PY{l+m+mi}{1}
             \PY{k}{for} \PY{n}{i} \PY{o+ow}{in} \PY{n+nb}{xrange}\PY{p}{(}\PY{n}{size}\PY{p}{)}\PY{p}{:}
                 \PY{n}{F}\PY{p}{[}\PY{n}{i}\PY{p}{]} \PY{o}{=} \PY{n}{a}
                 \PY{n}{a}\PY{p}{,} \PY{n}{b} \PY{o}{=} \PY{n}{b}\PY{p}{,} \PY{n}{a} \PY{o}{+} \PY{n}{b}
             \PY{k}{return} \PY{n}{F}
\end{Verbatim}

    and this works just fine:

    \begin{Verbatim}[commandchars=\\\{\}]
{\color{incolor}In [{\color{incolor}53}]:} \PY{n}{fibonacci}\PY{p}{(}\PY{l+m+mi}{20}\PY{p}{)}
\end{Verbatim}

            \begin{Verbatim}[commandchars=\\\{\}]
{\color{outcolor}Out[{\color{outcolor}53}]:} array([   0,    1,    1,    2,    3,    5,    8,   13,   21,   34,   55,
                  89,  144,  233,  377,  610,  987, 1597, 2584, 4181])
\end{Verbatim}
        
    However, what if we need one number at a time, or if we need a million
or 10 million values? In the first case, you would somehow have to store
the values from the last iteration, and restore the state to the
function each time it is called. In the second case, you would have to
generate and then store a very large number of values, most of which you
may not need right now.

A more sensible solution is to create a \texttt{generator}, which
calculates a single value in the sequence, then \emph{returns control
back to the caller}. This allows the generator to be called again,
resuming the sequence generation where it left off. Here's the Fibonacci
function, implemented as a generator:

    \begin{Verbatim}[commandchars=\\\{\}]
{\color{incolor}In [{\color{incolor}54}]:} \PY{k}{def} \PY{n+nf}{gfibonacci}\PY{p}{(}\PY{n}{size}\PY{p}{)}\PY{p}{:}
             \PY{n}{a}\PY{p}{,} \PY{n}{b} \PY{o}{=} \PY{l+m+mi}{0}\PY{p}{,} \PY{l+m+mi}{1}
             \PY{k}{for} \PY{n}{\PYZus{}} \PY{o+ow}{in} \PY{n+nb}{xrange}\PY{p}{(}\PY{n}{size}\PY{p}{)}\PY{p}{:}
                 \PY{k}{yield} \PY{n}{a}
                 \PY{n}{a}\PY{p}{,} \PY{n}{b} \PY{o}{=} \PY{n}{b}\PY{p}{,} \PY{n}{a} \PY{o}{+} \PY{n}{b}
\end{Verbatim}

    Notice that there is no \texttt{return} statement at all; just
\texttt{yield}, which is where a value is returned each time one is
requested. The \texttt{yield} statement is what defines a generator.

When we call our generator, rather than a sequence of Fibonacci numbers,
we get a generator object:

    \begin{Verbatim}[commandchars=\\\{\}]
{\color{incolor}In [{\color{incolor}55}]:} \PY{n}{f} \PY{o}{=} \PY{n}{gfibonacci}\PY{p}{(}\PY{l+m+mi}{100}\PY{p}{)}
         \PY{n}{f}
\end{Verbatim}

            \begin{Verbatim}[commandchars=\\\{\}]
{\color{outcolor}Out[{\color{outcolor}55}]:} <generator object gfibonacci at 0x10b88e640>
\end{Verbatim}
        
    A generator has a \texttt{\_\_next\_\_()} method that can be called
either via the method \texttt{generator.next()} or the builtin function
\texttt{next()}. The call to \texttt{next} executes the generator until
the \texttt{yield} statement is reached, returning the next generated
value, and then pausing until another call to \texttt{next} occurs:

    \begin{Verbatim}[commandchars=\\\{\}]
{\color{incolor}In [{\color{incolor}56}]:} \PY{n}{f}\PY{o}{.}\PY{n}{next}\PY{p}{(}\PY{p}{)}\PY{p}{,} \PY{n}{f}\PY{o}{.}\PY{n}{next}\PY{p}{(}\PY{p}{)}\PY{p}{,} \PY{n+nb}{next}\PY{p}{(}\PY{n}{f}\PY{p}{)}
\end{Verbatim}

            \begin{Verbatim}[commandchars=\\\{\}]
{\color{outcolor}Out[{\color{outcolor}56}]:} (0, 1, 1)
\end{Verbatim}
        
    A generator is a type of \texttt{iterator}. If we call a function that
supports iterables using a generator as an argument, it will know how to
use the generator.

    \begin{Verbatim}[commandchars=\\\{\}]
{\color{incolor}In [{\color{incolor}57}]:} \PY{n}{np}\PY{o}{.}\PY{n}{array}\PY{p}{(}\PY{n+nb}{list}\PY{p}{(}\PY{n}{f}\PY{p}{)}\PY{p}{)}
\end{Verbatim}

            \begin{Verbatim}[commandchars=\\\{\}]
{\color{outcolor}Out[{\color{outcolor}57}]:} array([2, 3, 5, 8, 13, 21, 34, 55, 89, 144, 233, 377, 610, 987, 1597, 2584,
                4181, 6765, 10946, 17711, 28657, 46368, 75025, 121393, 196418,
                317811, 514229, 832040, 1346269, 2178309, 3524578, 5702887, 9227465,
                14930352, 24157817, 39088169, 63245986, 102334155, 165580141,
                267914296, 433494437, 701408733, 1134903170, 1836311903, 2971215073,
                4807526976, 7778742049, 12586269025, 20365011074, 32951280099,
                53316291173, 86267571272, 139583862445, 225851433717, 365435296162,
                591286729879, 956722026041, 1548008755920, 2504730781961,
                4052739537881, 6557470319842, 10610209857723, 17167680177565,
                27777890035288, 44945570212853, 72723460248141, 117669030460994,
                190392490709135, 308061521170129, 498454011879264, 806515533049393,
                1304969544928657, 2111485077978050, 3416454622906707,
                5527939700884757, 8944394323791464, 14472334024676221,
                23416728348467685, 37889062373143906, 61305790721611591,
                99194853094755497, 160500643816367088, 259695496911122585,
                420196140727489673, 679891637638612258, 1100087778366101931,
                1779979416004714189, 2880067194370816120, 4660046610375530309,
                7540113804746346429, 12200160415121876738L, 19740274219868223167L,
                31940434634990099905L, 51680708854858323072L, 83621143489848422977L,
                135301852344706746049L, 218922995834555169026L], dtype=object)
\end{Verbatim}
        
    What happens when we reach the ``end'' of a generator?

    \begin{Verbatim}[commandchars=\\\{\}]
{\color{incolor}In [{\color{incolor}58}]:} \PY{n}{a\PYZus{}few\PYZus{}fibs} \PY{o}{=} \PY{n}{gfibonacci}\PY{p}{(}\PY{l+m+mi}{2}\PY{p}{)}
\end{Verbatim}

    \begin{Verbatim}[commandchars=\\\{\}]
{\color{incolor}In [{\color{incolor}59}]:} \PY{n}{a\PYZus{}few\PYZus{}fibs}\PY{o}{.}\PY{n}{next}\PY{p}{(}\PY{p}{)}
\end{Verbatim}

            \begin{Verbatim}[commandchars=\\\{\}]
{\color{outcolor}Out[{\color{outcolor}59}]:} 0
\end{Verbatim}
        
    \begin{Verbatim}[commandchars=\\\{\}]
{\color{incolor}In [{\color{incolor}60}]:} \PY{n}{a\PYZus{}few\PYZus{}fibs}\PY{o}{.}\PY{n}{next}\PY{p}{(}\PY{p}{)}
\end{Verbatim}

            \begin{Verbatim}[commandchars=\\\{\}]
{\color{outcolor}Out[{\color{outcolor}60}]:} 1
\end{Verbatim}
        
    \begin{Verbatim}[commandchars=\\\{\}]
{\color{incolor}In [{\color{incolor}61}]:} \PY{n}{a\PYZus{}few\PYZus{}fibs}\PY{o}{.}\PY{n}{next}\PY{p}{(}\PY{p}{)}
\end{Verbatim}

    \begin{Verbatim}[commandchars=\\\{\}]

        ---------------------------------------------------------------------------
    StopIteration                             Traceback (most recent call last)

        <ipython-input-61-0eaea96ec098> in <module>()
    ----> 1 a\_few\_fibs.next()
    

        StopIteration: 

    \end{Verbatim}

    Thus, generators signal when there are no further values to generate by
throwing a \texttt{StopIteration} exception. We must either handle this
exception, or create a generator that is infinite, which we can do in
this example by replacing a \texttt{for} loop with a \texttt{while}
loop:

    \begin{Verbatim}[commandchars=\\\{\}]
{\color{incolor}In [{\color{incolor}62}]:} \PY{k}{def} \PY{n+nf}{infinite\PYZus{}fib}\PY{p}{(}\PY{p}{)}\PY{p}{:}
             \PY{n}{a}\PY{p}{,} \PY{n}{b} \PY{o}{=} \PY{l+m+mi}{0}\PY{p}{,} \PY{l+m+mi}{1}
             \PY{k}{while} \PY{n+nb+bp}{True}\PY{p}{:}
                 \PY{k}{yield} \PY{n}{a}
                 \PY{n}{a}\PY{p}{,} \PY{n}{b} \PY{o}{=} \PY{n}{b}\PY{p}{,} \PY{n}{a} \PY{o}{+} \PY{n}{b}
\end{Verbatim}

    \begin{Verbatim}[commandchars=\\\{\}]
{\color{incolor}In [{\color{incolor}63}]:} \PY{n}{f} \PY{o}{=} \PY{n}{infinite\PYZus{}fib}\PY{p}{(}\PY{p}{)}
         \PY{n}{vals} \PY{o}{=} \PY{p}{[}\PY{n}{f}\PY{o}{.}\PY{n}{next}\PY{p}{(}\PY{p}{)} \PY{k}{for} \PY{n}{\PYZus{}} \PY{o+ow}{in} \PY{n+nb}{range}\PY{p}{(}\PY{l+m+mi}{10000}\PY{p}{)}\PY{p}{]}
         \PY{n}{vals}\PY{p}{[}\PY{o}{\PYZhy{}}\PY{l+m+mi}{1}\PY{p}{]}
\end{Verbatim}

            \begin{Verbatim}[commandchars=\\\{\}]
{\color{outcolor}Out[{\color{outcolor}63}]:} 20793608237133498072112648988642836825087036094015903119682945866528501423455686648927456034305226515591757343297190158010624794267250973176133810179902738038231789748346235556483191431591924532394420028067810320408724414693462849062668387083308048250920654493340878733226377580847446324873797603734794648258113858631550404081017260381202919943892370942852601647398213554479081823593715429566945149312993664846779090437799284773675379284270660175134664833266377698642012106891355791141872776934080803504956794094648292880566056364718187662668970758537383352677420835574155945658542003634765324541006121012446785689171494803262408602693091211601973938229446636049901531963286159699077880427720289235539329671877182915643419079186525118678856821600897520171070499437657067342400871083908811800976259727431820539554256869460815355918458253398234382360435762759823179896116748424269545924633204614137992850814352018738480923581553988990897151469406131695614497783720743461373756218685106856826090696339815490921253714537241866911604250597353747823733268178182198509240226955826416016690084749816072843582488613184829905383150180047844353751554201573833105521980998123833253261228689824051777846588461079790807828367132384798451794011076569057522158680378961532160858387223882974380483931929541222100800313580688585002598879566463221427820448492565073106595808837401648996423563386109782045634122467872921845606409174360635618216883812562321664442822952537577492715365321134204530686742435454505103269768144370118494906390254934942358904031509877369722437053383165360388595116980245927935225901537634925654872380877183008301074569444002426436414756905094535072804764684492105680024739914490555904391369218696387092918189246157103450387050229300603241611410707453960080170928277951834763216705242485820801423866526633816082921442883095463259080471819329201710147828025221385656340207489796317663278872207607791034431700112753558813478888727503825389066823098683355695718137867882982111710796422706778536913192342733364556727928018953989153106047379741280794091639429908796650294603536651238230626L
\end{Verbatim}
        
    \subsection{Exercise: Translate R to
Python}\label{exercise-translate-r-to-python}

Recode the secant search function from Bios 301 from R to Python.

    \begin{Verbatim}[commandchars=\\\{\}]
{\color{incolor}In [{\color{incolor}64}]:} \PY{o}{\PYZpc{}}\PY{k}{load} \PY{n}{http}\PY{p}{:}\PY{o}{/}\PY{o}{/}\PY{n}{git}\PY{o}{.}\PY{n}{io}\PY{o}{/}\PY{o}{\PYZhy{}}\PY{l+m+mi}{2}\PY{n}{DM8Q}
\end{Verbatim}

    \begin{Verbatim}[commandchars=\\\{\}]
{\color{incolor}In [{\color{incolor}65}]:} \PY{c}{\PYZsh{} Write your code here}
\end{Verbatim}

    \begin{Verbatim}[commandchars=\\\{\}]
{\color{incolor}In [{\color{incolor}66}]:} \PY{c}{\PYZsh{} Run this cell to load the answer}
         \PY{o}{\PYZpc{}}\PY{k}{load} \PY{n}{http}\PY{p}{:}\PY{o}{/}\PY{o}{/}\PY{n}{git}\PY{o}{.}\PY{n}{io}\PY{o}{/}\PY{n}{h5jqBg}
\end{Verbatim}

    How about a modified secant functiont that uses a generator?

    \begin{Verbatim}[commandchars=\\\{\}]
{\color{incolor}In [{\color{incolor}67}]:} \PY{c}{\PYZsh{} Write your code here}
\end{Verbatim}

    \begin{Verbatim}[commandchars=\\\{\}]
{\color{incolor}In [{\color{incolor}68}]:} \PY{c}{\PYZsh{} Run this cell to load the answer}
         \PY{o}{\PYZpc{}}\PY{k}{load} \PY{n}{http}\PY{p}{:}\PY{o}{/}\PY{o}{/}\PY{n}{git}\PY{o}{.}\PY{n}{io}\PY{o}{/}\PY{o}{\PYZhy{}}\PY{n}{CSooQ}
\end{Verbatim}

    \subsection{References}\label{references}

\begin{itemize}
\itemsep1pt\parskip0pt\parsep0pt
\item
  \href{http://learnpythonthehardway.org/book/}{Learn Python the Hard
  Way}\\
\item
  \href{http://learnxinyminutes.com/docs/python/}{Learn X in Y Minutes
  (where X=Python)}\\
\item
  \href{http://pythonforbiologists.com/index.php/29-common-beginner-python-errors-on-one-page/}{29
  common beginner Python errors on one page}
\item
  \href{http://www.jeffknupp.com/blog/2013/02/14/drastically-improve-your-python-understanding-pythons-execution-model/}{Understanding
  Python's Execution Model}
\end{itemize}

    \begin{center}\rule{3in}{0.4pt}\end{center}

    \begin{Verbatim}[commandchars=\\\{\}]
{\color{incolor}In [{\color{incolor}69}]:} \PY{k+kn}{from} \PY{n+nn}{IPython.core.display} \PY{k+kn}{import} \PY{n}{HTML}
         \PY{k}{def} \PY{n+nf}{css\PYZus{}styling}\PY{p}{(}\PY{p}{)}\PY{p}{:}
             \PY{n}{styles} \PY{o}{=} \PY{n+nb}{open}\PY{p}{(}\PY{l+s}{\PYZdq{}}\PY{l+s}{styles/custom.css}\PY{l+s}{\PYZdq{}}\PY{p}{,} \PY{l+s}{\PYZdq{}}\PY{l+s}{r}\PY{l+s}{\PYZdq{}}\PY{p}{)}\PY{o}{.}\PY{n}{read}\PY{p}{(}\PY{p}{)}
             \PY{k}{return} \PY{n}{HTML}\PY{p}{(}\PY{n}{styles}\PY{p}{)}
         \PY{n}{css\PYZus{}styling}\PY{p}{(}\PY{p}{)}
\end{Verbatim}

            \begin{Verbatim}[commandchars=\\\{\}]
{\color{outcolor}Out[{\color{outcolor}69}]:} <IPython.core.display.HTML at 0x10b895290>
\end{Verbatim}
        

    % Add a bibliography block to the postdoc
    
    
    
    \end{document}
